%!TEX program = xelatex
\documentclass{readarticle}

\begin{document}
  
\title{Title}
\author{Liudaohui}
\maketitle{}
\newpage
\section{论文笔记}

%论文基本信息
\subsection{Title}
    \begin{table}[htbp]
        \centering
        \small
        \caption{论文基本信息}
        \begin{tabular}{l|l}
            \toprule
            译名          & \\
            \midrule
            作者          & Author \\
            所在机构      & institute\\
            邮箱         & email \\
            出版年        & year \\
            文献类型      & type \\
            DOI          & DI \\
            期刊          & journal \\
            期刊分区      & \\
            期刊影响因子  & \\
            审稿周期     & month\\
            \bottomrule
        \end{tabular}%
        \label{tab:n.infomation}%
    \end{table}%
\subsection*{摘要}
\textcolor{blue}{abstract}
\newpage
%研究问题
\subsection{GAP}
    \begin{itemize}
        \item 1. 
    \end{itemize}

\subsection{我想从中了解到的内容}
    %作者所探究的关键技术问题以及带着什么问题去阅读?

    
    \begin{itemize}
        \item 1. 
    \end{itemize}


%作者的应用场景设定和方法
\subsection{作者的应用场景和方法}

    \begin{enumerate}[1.]
        \item  
    \end{enumerate}
%\par\noindent\rule{\textwidth}{0.4pt}%绘制横线代码

%作者的研究成果
\subsection{Result}
%作者实现的成果和技术
    \begin{enumerate}[1.]
        \item  
    \end{enumerate}

\subsection{Limitations}
%作者的工作中的一些缺陷或者可以优化的方面
    \begin{enumerate}[(1)]
        \item  
    \end{enumerate}


\subsection{我的总结体会}


\begin{enumerate}
    \item .
    \item .

\end{enumerate}
\begin{table}[htbp]
    \centering
    \small
    \caption{一些表达}
    \begin{tabular}{lll}
        \toprule
        英文 &中文 &备注\\
        \midrule
        & & \\
        \bottomrule
    \end{tabular}%
    \label{tab:n.keywords}%
\end{table}%


\end{document}
